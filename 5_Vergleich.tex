\section{Vergleich zwischen Half-Edge Mesh und UnityMesh}
Nachdem die Funktionsweise der wichtigsten Operationen f\"ur ein Half-Edge Mesh im vorherigen Kapitel beschrieben wurden, werden diese im Folgenden mit vergleichbaren Operationen eines UnityMeshs verglichen.

Bei den Methoden \textit{SplitHalfEdge}, \textit{EdgeCollapse} und \textit{SplitFace} handelt es sich um Operationen auf dem Mesh, die lokale \"Anderungen verursachen. Da es einen direkten Zugriff auf die betroffenen Stellen gibt, mithilfe der Face- und Half-Edge-Listen muss nicht das gesamte Mesh nach den betroffenen stellen abgesucht werden. Zudem kann mittels der \textit{GetFaceCirculator}-Methode auf alle betroffenen Half-Edges in konstanter Zeit zugegriffen werden, da diese Methode immer die drei Half-Edges \"uber die Referenz der Face bestimmen kann. Insgesamt haben die \textit{SplitHalfEdge}- und \textit{SplitFace}-Methode also eine konstante Zeitkomplexit\"at, unabh\"angig von der gr\"o{\ss}e des Meshs.
Um allerdings das gleiche Resultat mit einem UnityMesh zu erhalten, muss zuerst mittels Brute-Force jeder Index der Eckpunkte der gesuchten Kante oder Face ermittelt werden, da diese nicht explizit dargestellt werden. Diese Suche nach allen Kanten mit den selben Eckpunkten k\"onnte wie folgt aussehen:
\begin{lstlisting}
List<int> findEdges(int a, int b)
{
	// --- All triangle indices
	int[] tris = _mesh.triangles;
	
	// --- The result contains every starting index 
	// --- of a matching triangle
	List<int> result = new List<int>();
	
	// --- for each pair of three
	for(int i = 0; i < tris.Length; i+=3)
	{
		int matches = 0; // --- count the matching indices
		if(tris[i] == a || tris[i] == b)
			matches++;
		if(tris[i+1] == a || tris[i+1] == b)
			matches++;
		if(tris[i+2] == a || tris[i+2] == b)
			matches++;
			
		// --- if there are two, consider this triangle index
		if(matches == 2) 
			result.Add(i);
	}
	return result;
}
\end{lstlisting}
Dieser beispielhafte Implementierungsvorschlag kann benutzt werden, um die betroffenen Faces f\"ur einen EdgeSplit zu finden, um die entsprechende Stelle f\"ur eine Subdivision zu finden, muss der dritte betroffene Punkt mitgegeben werden. Auff\"allig bei dieser Implementierung ist, dass es, im Gegensatz zum Half-Edge-Mesh abh\"angig von der gr\"o{\ss}e des Meshs ist, wodurch es bei dieser Implementierung mit gro{\ss}en Netzen zu Laufzeitproblemen kommen kann.

Auch die \textit{EdgeCollapse}-Methode verwendet die \textit{GetFaceCirculator}-Methode mit konstanter Laufzeit. Zus\"atzlich wird die \textit{GetVertexCirculator}-Methode verwendet, um die Half-Edges an den betroffenen Vertices zu finden. Theoretisch besitzt diese Methode eine konstante Laufzeit in Abh\"angigkeit von der Anzahl der Half-Edges, die den gefragten Vertex verwenden. Im Plankton-Mesh \cite{Meshmash2017} wird folgende Implementierung verwendet:
\begin{lstlisting}
public IEnumerable<int> GetVertexCirculator(int halfedgeIndex)
{
	if (halfedgeIndex < 0 || halfedgeIndex > this.Count) { 
		yield break; 
	}
	
	int h = halfedgeIndex;
	int count = 0;
	do
	{
		yield return h;
		h = this[this.GetPairHalfedge(h)].NextHalfedge;
		if (h < 0) { 
		throw new InvalidOperationException
			("Unset index, cannot continue."); 
		}
		if (count++ > 999) {
			throw new InvalidOperationException
				("Runaway vertex circulator"); 
		}
	}
	while (h != halfedgeIndex);
}
\end{lstlisting}
Allerdings funktioniert dieser Ansatz nur auf der zugrunde liegenden Annahme, dass das abgebildete Mesh vollst\"andig ist, es also keine Half-Edge gibt, die keinen Nachbarn hat. Aus diesem Grund wird eine andere Implementierung verwendet, da diese Eigenschaft f\"ur ein solches Netz nicht gegeben sein muss. Diese sieht wie folgt aus: 
\begin{lstlisting}
public List<HalfEdge> GetVertexCirculator(Vertex v)
{
	return _mesh.HalfEdges
	.Where(p => p.OutgoingPoint.Index == v.Index).ToList();
}
\end{lstlisting}
Durch diese Generalisierung wird allerdings die konstante Laufzeit durch eine Lineare ersetzt, wodurch die Laufzeiten des Half-Edge-Meshs und des UnityMeshs jeweils in lineare Abh\"angigkeit von der gr\"o{\ss}e des Netzes stehen. Allerdings kann, wie das Plankton-Mesh zeigt, die \textit{GetVertexCirculator}-Methode noch weiter optimiert werden. 

Alles in allem kann gesagt werden, dass ein Half-Edge-Mesh mit konstanten Laufzeiten f\"ur die wichtigsten Operationen besser geeignet ist f\"ur Echtzeitsimulationen mit dynamischen Meshs, die sich zur Laufzeit ver\"andern. Auch eignet sich ein Half-Edge-Mesh zur Bearbeitung eines Meshs, um den Detailgrad mithilfe der Subdivision zu erh\"ohen oder durch EdgeCollapse zu reduzieren. Allerdings kommen diese M\"oglichkeiten mit dem Nachteil, dass die Grundstruktur des Half-Edge-Mesh allein mehr als sechs mal so gro{\ss} ist, wie ein Indexed Mesh der selben Anzahl an Vertices. F\"ur statische Meshs eignet sich daher eher das UnityMesh und allgemein muss auf den einzelnen Anwendungsfall geachtet werden, um zu entscheiden, ob ein Half-Edge-Mesh oder ein Indexed Mesh verwendet werden soll.

\section{Ausblick}
Im Anschluss an diese Arbeit gibt es noch einige Features, um die diese Implementierung eines Half-Edge-Mesh erweitert werden kann. Zum Einen muss die \textit{GetVertexCirculator}-Methode optimiert werden, um eine konstante Zeitkomplexit\"at bei allen Grundoperationen zu erhalten, damit es in Echtzeitanwendungen verwendet werden kann. 

Des Weiteren ist ein Ziel, die 2D-Zerst\"orungssimulation zu erweitern, um diese in dreidimensionalen Simulationen verwenden zu k\"onnen. Dies kann geschehen, indem die Bruchpunkte nicht nur, wie bei der 2D-Simulation, auf der Oberfl\"ache, sondern auch innerhalb des Modells verteilt werden. Die erzeugten ,,Scherben'' werden um einen vierten Punkt erweitert, wodurch diese ein Tetraeder bilden und zu einem dreidimensionalen Splitter werden. 

In diesem Zuge sollte die Simulation auch die Kraft (force), die auf das Modell wirkt, betrachtet werden, um in unterschiedlichen Situationen realistischere Resultate zu erzielen.

Sollte der gew\"ahlte Ansatz nicht die gew\"unschten Effekte erzielen, kann eine andere Methode verwendet werden, um die Brucheffekte zu berechnen, zum Beispiel mithilfe von Voronoi-Diagrammen. 

Zum Schluss soll alles in einer Simulation zusammengef\"ugt werden, die als Demo dient, in der die Effekte gezeigt werden, um sie anschlie{\ss}end in Anwendungen, wie zum Beispiel Computerspielen verwenden zu k\"onnen.