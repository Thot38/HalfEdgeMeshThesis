\section{Half-Edge-Mesh}
Um die oben genannten Problemstellungen zu l\"osen, gibt es andere Ans\"atze Polygonalnetze zu realisieren. Einer dieser L\"osungen ist das Konzept der Half-Edge-Meshes. Ein solches Mesh besteht aus folgenden Komponenten: 
\begin{itemize}
	\item eine Liste von Vertices
	\item eine Liste von Faces
	\item eine Liste von Half-Edges.
\end{itemize}

Die Vertices sind, wie im Unity-Mesh auch, die Eckpunkte der Polygonen. Die Vertices werden mithilfe von Half-Edges miteinander verbunden. 

\subsection{Vertex}
Ein Vertex ist ein Punkt im dreidimensionalen Raum und ist wie beim einfachen Mesh der Eckpunkt f\"ur die Polygonen. Neben einem \textit{Vector3} f\"ur die Position des Vertex wird eine Referenz auf eine ausgehende Half-Edge, sowie der Index des Vertex ben\"otigt.
\\
\begin{lstlisting}
public class Vertex
{
	public event EventHandler PositionChangedEvent;
	
	public Vertex(Vector3 point)
	{
		Point = point;
		Index = -1;
	}

	public Vertex(Vector3 point, int index)
	{
		Point = point;
		Index = index;
	}

	public Vertex(float x, float y, float z) : this(new Vector3(x, y, z))
	{ }

	public Vertex(float x, float y, float z, int index) : this(new Vector3(x, y, z), index)
	{ }

	/// <summary>
	/// The Index of the Vertex
	/// </summary>
	public int Index { get; set; }

	private Vector3 _point;
	/// <summary>
	/// The Point
	/// </summary>
	public Vector3 Point
	{
		get => _point;
		set
		{
			_point = value;
			PositionChangedEvent?.Invoke(this, EventArgs.Empty);
		}
	}

	/// <summary>
	/// The Outgoing HalfEdge
	/// </summary>
	public HalfEdge HalfEdge { get; set; }
}
\end{lstlisting}

\subsection{Face}
Ein Objekt der Klasse Face repr\"asentiert die Faces des Meshs, die \"uber die Vertices aufgespannt werden. Dabei besitzt eine Face nur die Referenz auf eine der anliegenden Half-Edges um das Traversieren und Bearbeiten des Netzes auf Face-Ebene zu erm\"oglichen. Die Face-Klasse ist wie folgt aufgebaut:
\begin{lstlisting}
public class Face
{
	public Face(HalfEdge halfEdge)
	{
		HalfEdge = halfEdge;
	}

	/// <summary>
	/// A bonding HalfEdge
	/// </summary>
	public HalfEdge HalfEdge { get; set; }
}
\end{lstlisting}

\subsection{Half-Edge}
Die Half-Edge Klasse ist die wichtigste Komponente f\"ur das Half-Edge-Mesh. Sie stellt alle Teile in Beziehung zueinander. Zu bemerken ist, dass ein Half-Edge-Mesh keine explizite Modellierung von Edges hat. Eine Edge wird implizit durch zwei Half-Edges abgebildet. Jede Half-Edge besitzt eine Referenz auf ihre gegen\"uberliegende Half-Edge, sowie auf die ihr Nachfolgende. Zudem referenziert jede Half-Edge den Vertex, aus dem sie hervorgeht und die Face, die sie einschlie{\ss}t. 
\begin{lstlisting}
public class HalfEdge
{
	public HalfEdge(Vertex outgoing, Face face, HalfEdge next)
	{
		OutgoingPoint = outgoing;
		Face = face;
		Next = next;
		Index = -1;
	}
	
	public HalfEdge(Vertex outgoing, Face face, HalfEdge next, int index)
	{
		OutgoingPoint = outgoing;
		Face = face;
		Next = next;
		Index = index;
	}
	
	public int Index { get; set; }
	
	/// <summary>
	/// The Vertex the HalfEdge comes from
	/// </summary>
	public Vertex OutgoingPoint { get; set; }
	
	/// <summary>
	/// The next HalfEdge, Counter Clockwise
	/// </summary>
	public HalfEdge Next { get; set; }
	
	/// <summary>
	/// The opposing HalfEdge
	/// </summary>
	public HalfEdge Opposing { get; set; }
	
	/// <summary>
	/// The face of the HalfEdge
	/// </summary>
	public Face Face { get; set; }
	
	/// <summary>
	/// Getter for the previous HalfEdge, for easier access 
	/// (this.Next.Next)
	/// </summary>
	public HalfEdge Previous => Next.Next;
	
	/// <summary>
	/// Getter for the EndPoint of this HalfEdge, for easier access (this.Next.OutgoingPoint)
	/// </summary>
	public Vertex EndPoint => Next.OutgoingPoint;
}
\end{lstlisting} 
