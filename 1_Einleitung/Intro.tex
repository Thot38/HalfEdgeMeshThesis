\section{Abstract}
Das Ziel dieser Arbeit ist es, die Performance von Unity-Meshs mit der eines Half-Edge Mesh zu vergleichen. Im folgenden wird versucht, die folgende Forschungsfrage zu beantworten: Wie unterscheidet sich die Laufzeit von Unity-Meshs und Half-Edge Meshs, bei der Ausf\"uhrung von Standartmethoden f\"ur dynamische Anwendungen und was bedeutet dies f\"ur ihre Anwendungsbereiche?
Um diese Frage zu beantworten, wird das Half-Edge Mesh theoretisch betrachtet, um Aussagen \"uber das durchschnittliche Laufzeitverhalten treffen zu k\"onnen und einen Prototypen zu implementieren, auf dem diverse Benchmark-Tests durchgef\"uhrt werden.
Diese Analysen haben gezeigt, dass sich Half-Edge Meshs durch ihre deutlich bessere Laufzeit f\"ur dynamische Meshs besser eignen, daf\"ur allerdings einen h\"oheren Speicherbedarf aufweisen. Deshalb sollte bei gr\"o{\ss}eren Meshs abgewogen werden, ob der Anwendungsfall ein Half-Edge Mesh ben\"otigt oder ob die native Unity-L\"osung ausreichend ist.