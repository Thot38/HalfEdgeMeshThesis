\section{Einleitung}
In der Computergrafik werden dreidimensionale Modelle als Polygonen-Netz (Polygonal Mesh) dargestellt, um diese auf einem zweidimensionalen Bildschirm darzustellen. Die Oberfl\"ache eines Modells wird dabei mit Hilfe von Polygonen angen\"ahert. H\"aufig werden daf\"ur Dreiecks-Netze (Triangle Mesh) verwendet, wobei die Fl\"achen mit Dreiecken nachmodelliert werden, siehe Abbildung \ref{fig:dolphintrianglemesh}. 
\\Ein solches Netz besteht aus den Eckpunkten der einzelnen Dreiecke, den Vertices. Diese werden durch Kanten (Edges) verbunden und bilden damit die Polygonalfl\"achen, auch Faces genannt. 

\begin{figure}[h]
	\centering
	\includegraphics[width=0.7\linewidth]{Images/Dolphin_triangle_mesh}
	\caption[Beispiel eines Polygonen-Netzes]{Beispiel eines Dreiecks-Polygonen-Netz, von \cite{WikipediaDolphin}}
	\label{fig:dolphintrianglemesh}
\end{figure}

\section{Unity}
Unity ist eine Spiele-Engine mit eingebauter Entwicklungsumgebung f\"ur 2D-, 3D- und VR-Spiele/-Simulationen. Die Engine kommt mit einem eigenen Editor, in welchem diverse Szenarien erstellt und bearbeitet werden k\"onnen. Des Weiteren unterst\"utzt Unity selbst programmierte Scripte auf der Grundlage von C\#.
\subsection{Meshes in Unity}
Unity bietet die M\"oglichkeit, mit Hilfe von selbstgeschriebenen Scripten eigene 3D-Modelle zur Laufzeit erstellen zu lassen. Daf\"ur stellt Unity ein eigenes Mesh-System zur Verf\"ugung, basierend auf Dreiecksnetzen, die \textit{UnityEngine.Mesh}-Klasse. Damit diese ein Mesh rendern kann, erwartet das Mesh ein \textit{UnityEngine.Vector3}-Array f\"ur die Vertices, wobei ein \textit{Vector3} einen Punkt im dreidimensionalen Raum darstellt und ein \textit{int}-Array, das die Reihenfolge der Vertices f\"ur die Dreiecke festlegt.   
\\
Der folgende Code zeigt beispielhaft, wie ein Unity-Mesh erzeugt werden kann:
\begin{lstlisting}
public void CreateMesh()
{
	//--- Der Vollstaendigkeit halber vorhanden
	meshFilter = gameObject.GetComponent<MeshFilter>();
	if (meshFilter == null)
		meshFilter = gameObject.AddComponent<MeshFilter>();

	//--- vom MeshFilter zum Mesh
	mesh = meshFilter.sharedMesh;
	if (mesh == null)
		mesh = new Mesh { name = "Quad" };

	//--- MeshRenderer holen
	meshRenderer = this.gameObject.GetComponent<MeshRenderer>();
	if (meshRenderer == null)
		meshRenderer = gameObject.AddComponent<MeshRenderer>();

	//--- Mesh zusammenstellen
	//--- Vertices/Points
	var P0 = new Vector3(0, 0, 0);
	var P1 = new Vector3(0, 1, 0);
	var P2 = new Vector3(1, 0, 0);
	var P3 = new Vector3(1, 1, 0);
	
	var verticies = new List<Vector3> { P0, P1, P2, P3 };

	//--- Triangles
	var triangles = new List<int>();

	triangles.Add(0);
	triangles.Add(1);
	triangles.Add(2);
	triangles.Add(2);
	triangles.Add(1);
	triangles.Add(3);

	//--- Mesh befuellen
	mesh.Clear();
	//--- Vertices zuweisen
	mesh.vertices = verticies.ToArray();
	//--- Triangles zuweisen
	mesh.triangles = triangles.ToArray();
	//--- Mesh dem MeshFilter zuweisen
	meshFilter.sharedMesh = mesh;
}
\end{lstlisting}

Und erzeugt folgendes Ergebnis:
\begin{figure}[h]
	\centering
	\includegraphics[width=0.35\linewidth]{Images/UnityQuadWireframe}
	\caption[Die Wireframeansicht des erstellten Meshes]{Die Wireframeansicht des erstellten Meshes}
	\label{fig:unityquadwireframe}
\end{figure}


